\documentclass{article}
\begin{document}

\title{Shreenabh Agrawal's Daily Challenge 7}
\author{Jose Bedoya}
\maketitle
{\normalsize
\vspace{10mm}

Prove that if 

$$a = 2+2\sqrt{28n^2+1}$$

is an integer, then it is a perfect square.

\vspace{7mm}
\textbf{Solution}

\vspace{5mm}
Notice that for $a$ to be an integer

$$28n^2+1$$

must be a perfect square.
Therefore,
$$28n^2+1 = q^2, \, \mbox{for some} \, q \in \textbf{N}$$

It's easy to see that $q$ is an odd number. Hence,

$$7n^2= \left(\frac{q+1}{2}\right)\left(\frac{q-1}{2}\right)$$

As these are two consecutive numbers, they are prime relatives (no common divisors). This is, they cannot form a perfect square if not by separate. In addition, as $7$ is prime, it must divide just one of them. Therefore,

$$\frac{q+1}{2}= 7r^2,\hspace{3mm} \frac{q-1}{2}=s^2$$
or
\vspace{2mm}
$$\frac{q+1}{2}= s^2,\hspace{3mm} \frac{q-1}{2}=7r^2$$
for some $s, r \in \textbf{N}$.

\vspace{3mm}
Notice the first case is not possible as $s^2 \equiv -1$ (mod $7$)
In other words, there are no perfect squares congruent to $-1$ (mod $7$). This is trivial to see, just make a list of numbers from $1$ to $6$, square each of them and you will see no final number has remainder of $6$ (same as saying $-1$) when divided by $7$.

\vspace{5mm} 
Hence,

$$a = 2+2\sqrt{28n^2+1} = 2(q+1)$$
$$\Rightarrow a = 4s^2 = (2s)^2$$

which is always a perfect square.
}
\end{document}