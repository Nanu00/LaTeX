\documentclass{article}
\usepackage[utf8]{inputenc}
\usepackage{amsmath}
\usepackage{amssymb}
\usepackage[legalpaper, portrait, margin=1in]{geometry}

\title{Mathinity Factorial Problem}

\author{Shreenabh Agrawal}

\date{\today}

\begin{document}

\maketitle

\section{Question}
$$Let \: S_{n}=1 !+2 !+\ldots+n !$$
Find all values of $n$ such that $S_{n}$ is a perfect square and prove that these are the only ones.
\section{Solution}
Let us start by calculating some values of this summation:
$$\begin{aligned}
S_{1} &=1 ! \\
&=1 \\
S_{2} &=1 !+2 ! \\
&=1+2 \\
&=3 \\
S_{3} &=1 !+2 !+3 ! \\
&=1+2+6 \\
&=9 \\
S_{4} &=1 !+2 !+3 !+4 !\\
&=1+2+6+24 \\
&=33 
\end{aligned}$$
Here, we can easily find the trivial solutions $$n = \: 1 \: \& \: n = \: 3$$
Next, from $S_5$ onwards, there will always be multiples of 10 that would be added to $S_4$ (because $5!$ and consequently all larger factorials have 10 as a factor). Hence, we can write as (for $n > 4$)

$$\begin{aligned}
S_{n} & \equiv\left(S_{4}+10 k\right) \bmod 10 \\
& \equiv S_{4} \bmod 10 \\
& \equiv 3 \bmod 10
\end{aligned}$$
Which as we know can never be perfect squares. Thus our only answers are:

$$\boxed{n = 1, \: 3}$$
\end{document}
