\documentclass[12pt]{article}
\usepackage[utf8]{inputenc}
\usepackage{xcolor}
\usepackage[T1]{fontenc}
\usepackage{pagecolor}
\usepackage{amssymb}
\usepackage{lmodern}
\usepackage{mathtools, nccmath}
\usepackage{courier}
\usepackage[overload]{empheq}
\usepackage[inline, shortlabels]{enumitem}
\usepackage{amsmath}
\usepackage{mathtools} 
\color{white}
\title{Solution to Number Theory \# 1}
\author{@all.about.mathematics}


\pagecolor{black}
\begin{document}
\maketitle
\section{Problem}
Let $S\subset\mathbb{N}$ such that $|S|=n$. Prove that $\exists A\subseteq S$ such that the sum of all elements in $A$ are divisible by $n$.
\section{Solution}
Let $S=\{a_1,a_2,\cdots,a_n\}$. 
Then define the following sums: $$S_1=a_1$$
$$S_2=a_1+a_2$$
$$\vdots$$
$$S_n=a_1+a_2+\cdots+a_n$$
\subsection{Case 1}
If one of the above sums are divisible by $n$, then the claim is proved in this case.
\subsection{Case 2}
If none of the sums are divisible by $n$, then, by the Pigeonhole Principle, at least 2 of the sums must have the same remainder when divided by $n$. This implies that we can pick 2 sums $S_b$ and $S_c$  with $b>c$ such that 
$$S_b-S_c=a_c+1+\cdots+a_b\equiv0(mod\:n)$$
Therefore the claim is proved in this case.

\end{document}