\documentclass[12pt]{article}
\usepackage[utf8]{inputenc}
\usepackage{xcolor}
\usepackage[T1]{fontenc}
\usepackage{pagecolor}
\usepackage{amssymb}
\usepackage{lmodern}
\usepackage{mathtools, nccmath}
\usepackage{courier}
\usepackage[overload]{empheq}
\usepackage[inline, shortlabels]{enumitem}
\usepackage{amsmath}
\usepackage{mathtools} 
\color{white}
\title{Solution to Number theory \#2}
\author{@all.about.mathematics}


\pagecolor{black}
\begin{document}
\maketitle
\section{Problem}
\noindent\fbox{%
    \parbox{\textwidth}{%
        Is there a power of 2 ($>8$) such that that its digits can be rearranged to form another power of 2? (zeros are not allowed in leading digits, e.g. 032 is not allowed)
    }%
}
\section{solution}
We list out some simple and obvious facts here to help us with the problem:

1. Rearranging digits does not change the total number of digits.

2. At most 4 consecutive powers of 2 can have the same number of digits.

3. A number is equal to its digit sum modulo 9.

\noindent{Using these, we rephrase the question:}

\noindent\fbox{%
    \parbox{\textwidth}{%
    Is there a power of 2 such that there is another power of 2 with the same number of digits and the same remainder modulo 9?
    }%
}

Observing the remainders modulo 9 of some powers of 2, we conjecture that the remainders repeat themselves every 6 powers.
$$\because 64\equiv1 \mod 9$$
$$\therefore 2^{n+6}\equiv2^n\times 64\equiv 2^n \mod 9$$
Since the remainders modulo 9 repeat themselves every 6 powers and there can only be at most 4 powers of 2 can have the same digits, the answer to the question is NO.
\end{document}