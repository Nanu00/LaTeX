\documentclass{article}
\usepackage[utf8]{inputenc}
\usepackage{amsmath}
\usepackage{xcolor}
\usepackage{mathptmx}
\usepackage{t1enc}
\usepackage{tikz}
\usetikzlibrary{arrows}
\usetikzlibrary{decorations.markings}
\pagecolor{black}
\color{white}

\begin{document}
\begin{center}
 \Huge
 \begin{equation}
  \int\tan{(x)} \text{ $dx$} \nonumber   
 \end{equation}

 \end{center}
\begin{center}
\newline
\newline
 \Huge Swipe For Solution. 
 \end{center}
\begin{itemize}
\centering \color{red}\Huge \item $\Rightarrow$  
\end{itemize}
\newpage
\normalsize
We can make this easier solve by breaking down tangent of x into functions we can work with.
\begin{eqnarray}
\sin{x} &=& \frac{opp}{hyp}\nonumber \\
\cos{x} &=& \frac{adj}{hyp} \nonumber
\end{eqnarray}
If we take the ratio between sine and cosine of x we see that $\frac{\sin{x}}{\cos{x}}$ is equal to $\frac{\frac{opp}{hyp}}{\frac{adj}{hyp}}$ which by further simplification is equivalent to $\frac{opp}{adj}$ which is why
\begin{equation}
    \frac{\sin{x}}{\cos{x}} = \frac{\frac{opp}{hyp}}{\frac{adj}{hyp}} = \tan{x} \nonumber
\end{equation}
Therefore our integral is equivalent to $\int\frac{\sin{x}}{\cos{x}}dx$
\newpage
\begin{center}
$\int\tan{(x)} dx$=$\int\frac{\sin{x}}{\cos{x}}dx$    
\end{center}
We can solve this integral with a simple u sub.
Let u be equal to $\cos{x}$ then the following is obvious.

\begin{eqnarray}
du &=& -\sin{x}dx \nonumber \\
-du & = & \sin{x}dx \nonumber
\end{eqnarray}
\textit{Our integral becomes}
\begin{eqnarray}
\int\tan{(x)} = -1\int\frac{1}{u}du \nonumber \\
\int\tan{(x)} = -1\ln{|u|} \nonumber
\end{eqnarray}
\textit{according to the laws of logarithms we can change our integral to}
\begin{eqnarray}
\int\tan{(x)} = \ln{|(u)^-1|} \nonumber\\
\int\tan{(x)} = \ln{|\frac{1}{u}|}\nonumber\\
\int\tan{(x)} = \ln{|\frac{1}{\cos{x}}|}\nonumber\\
\int\tan{(x)} =  \ln{|\sec{x}|}+c \nonumber
\end{eqnarray}
Almost forgot the + c :0
\newpage
\Huge
\begin{equation}
 \int\tan{(x)} =  \ln{|\sec{x}|}+c \nonumber
\end{equation}





\end{document}