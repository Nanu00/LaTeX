\documentclass{article}
\usepackage[utf8]{inputenc}
\usepackage{amsmath}
\usepackage[most]{tcolorbox}
\usepackage{booktabs}
\usepackage{xparse}
\usepackage{tikz}
\usetikzlibrary{calc}
\usepackage{pstricks,pst-node}
\pagestyle{empty}
\newcommand{\diff}[1]{\mathrm{d}#1}
\newcommand{\me}{\mathrm{e}}
\tikzset{Arrow Style/.style={text=black, font=\boldmath}}
\usepackage{cancel}
\newcommand{\tikzmark}[1]{%
    \tikz[overlay, remember picture, baseline] \node (#1) {};%
}

\newcommand*{\XShift}{0.5em}
\newcommand*{\YShift}{0.5ex}

\NewDocumentCommand{\DrawArrow}{s O{} m m m}{%
    \begin{tikzpicture}[overlay,remember picture]
        \draw[->, thick, Arrow Style, #2] 
                ($(#3.west)+(\XShift,\YShift)$) -- 
                ($(#4.east)+(-\XShift,\YShift)$)
        node [midway,above] {#5};
    \end{tikzpicture}%
}


\tcbset{colback=blue!10!white, colframe=blue!50!black, 
        highlight math style= {enhanced, %<-- needed for the ’remember’ options
            colframe=red,colback=red!10!white,boxsep=0pt}
        }
\begin{document}

\begin{equation*}
    I=\int_{0}^{\infty} \frac{\ln{x}}{(1+x^2)(1+x^3)}\,dx
\end{equation*}
This may be a long one, but let's get into it. We break up the bounds of integration. 
\begin{equation*}
    I=\int_{0}^{1} \frac{\ln{x}}{(1+x^2)(1+x^3)}\,dx+\int_{1}^{\infty} \frac{\ln{x}}{(1+x^2)(1+x^3)}\,dx
\end{equation*}
Now we will introduce a substitution $x \rightarrow 1/x$ into the second integral,
\begin{equation*}
    \begin{split}
       &= \int_{0}^{1} \frac{\ln{x}}{(1+x^2)(1+x^3)}\,dx+\int_{1}^{0} \frac{\ln(1/x)}{(1+\left(1/x\right)^2)(1+\left(1/x\right)^3)}\frac{-dx}{x^2} \\
        &=\int_{0}^{1} \frac{\ln{x}}{(1+x^2)(1+x^3)}\,dx-\int_{0}^{1} \frac{x^3\ln{x}}{(1+x^2)(1+x^3)}\,dx \\
        &=\int_{0}^{1} \frac{\ln{x}(1-x^3)}{(1+x^2)(1+x^3)}\,dx
    \end{split}
    
\end{equation*}
We can now perform partial fractional decompostion on the rational multiple of $\ln{x}$ to show that
\begin{equation*}
   I=\underbrace{\int_{0}^{1}\frac{x\ln{x}}{1+x^2}\,dx}_{I_1}-\frac{2}{3}\underbrace{\int_{0}^{1} \frac{\ln{x}(2x-1)}{x^2-x+1}\,dx}_{I_2}+\frac{1}{3}\underbrace{\int_{0}^{1}\frac{\ln{x}}{1+x}\,dx}_{I_3} 
\end{equation*}
So,
\begin{equation}
    \boxed{I=I_1-\frac{2}{3}I_2+\frac{1}{3}I_3}
\end{equation} \newpage
Let's begin by solving $I_1$,
\begin{equation*}
    I_1=\int_{0}^{1}\frac{x\ln{x}}{1+x^2}\,dx
\end{equation*}
We introduce the substitution $u=x^2$,
\begin{equation*}
\begin{split}
    I_1=\frac{1}{2}\int_{0}^{1} \frac{\ln{\sqrt{u}}}{1+u}\,du
    \\\frac{1}{4}\int_{0}^{1} \frac{\ln{u}}{1+u}&=\end{split}
\end{equation*}
We can now integrate by parts,


\[
    \renewcommand{\arraystretch}{1.5}
    \begin{array}{c @{\hspace*{1.0cm}} c}\toprule
       D & I \\\cmidrule{1-2}
      +\ln{u}\tikzmark{Left 1} & \tikzmark{Right 1}\frac{1}{1+u} \\
      -\frac{1}{u} \tikzmark{Left 2} & \tikzmark{Right 2}\ln(1+u)  \\\bottomrule
    \end{array}
\]
%-----------------------------------------
\DrawArrow[draw=blue]{Left 1}{Right 2}{$ $}%
\DrawArrow[draw=red]{Left 2}{Right 2}{$ $}%
to get \begin{equation*}
    \begin{split}
        I_1&= \underbrace{\frac{1}{4}\ln{u}\ln(1+u)\Big|_{0}^{1}}_{=0}-\frac{1}{4}\int_{0}^{1}\frac{\ln(1+u)}{u}\,du
    \end{split}
\end{equation*}
We now let $u=-x$, and have
\begin{equation*}
    \begin{split}
        &=\frac{1}{4}\int_{0}^{-1}\frac{\ln(1-x)}{-x}\,dx\\
        &=\frac{1}{4}\int_{-1}^{0}\frac{\ln(1-x)}{x}\,dx\\
        &=\frac{1}{4}\text{Li}_2(-1)=\frac{1}{4}\left(-\frac{\pi^2}{12}\right)=-\frac{\pi^2}{48}
    \end{split}
\end{equation*} \newpage
So,
\begin{equation}
    \boxed{I_1=-\frac{\pi^2}{48}}
\end{equation}
Now we must calculate $I_2$, which will be slightly more difficult.
\begin{equation*}
    I_2=\int_{0}^{1} \frac{\ln{x}(2x-1)}{x^2-x+1}\,dx
\end{equation*}
We now once again integrate by parts
\[
    \renewcommand{\arraystretch}{1.5}
    \begin{array}{c @{\hspace*{1.0cm}} c}\toprule
       D & I \\\cmidrule{1-2}
      +\ln{x}\tikzmark{Left 1} & \tikzmark{Right 1}\frac{2x-1}{x^2-x+1} \\
     - \frac{1}{x} \tikzmark{Left 2} & \tikzmark{Right 2}\ln(x^2-x+1)  \\\bottomrule
    \end{array}
\]
%-----------------------------------------
\DrawArrow[draw=blue]{Left 1}{Right 2}{$ $}%
\DrawArrow[draw=red]{Left 2}{Right 2}{$ $}%
\begin{equation*}
    \begin{split}
        I_2&=\underbrace{\ln{x}\ln(x^2-x+1)\Big|_{0}^{1}}_{=0}-\int_{0}^{1} \frac{\ln(x^2-x+1)}{x}\,dx \\
        &=-\int_{0}^{1} \frac{\ln(x^2-x+1)}{x}\,dx 
    \end{split}
\end{equation*}
Lets now factorize the polynomial argument of the logarithm and see what we can do. We notice that
\begin{equation*}
    x^2-x+1=0 \implies x=\frac{1}{2}\pm\frac{\sqrt{3}}{2}=e^{\pm i \frac{\pi}{3}}
\end{equation*}
So we can factorize the polynomial
\begin{equation*}
     x^2-x+1=0=(x-e^{i\frac{\pi}{3}})(x-e^{-i\frac{\pi}{3}})
\end{equation*} \newpage
So our integral becomes
\begin{equation*}
    \begin{split}
        I_2&=-\int_{0}^{1} \frac{\ln((x-e^{i\frac{\pi}{3}})(x-e^{-i\frac{\pi}{3}}))}{x}\,dx \\
        &=-\int_{0}^{1} \frac{\ln(x-e^{i\frac{\pi}{3}})}{x}\,dx-\int_{0}^{1} \frac{\ln(x-e^{-i\frac{\pi}{3}})}{x}\,dx \\ \\
    \end{split}
\end{equation*}
 Let's simplify with some algebra and log properties.
    \begin{equation*}
        \begin{split}
        &=-\int_{0}^{1} \frac{\ln(-e^{i\frac{\pi}{3}}(-e^{-i\frac{\pi}{3}}x+1))}{x}\,dx-\int_{0}^{1} \frac{\ln(-e^{-i\frac{\pi}{3}}(-e^{i\frac{\pi}{3}}x+1))}{x}\,dx\\
        &=-\int_{0}^{1} \frac{\ln(-e^{-i\frac{\pi}{3}}x+1)}{x}\,dx  -\int_{0}^{1} \frac{\ln(-e^{i\frac{\pi}{3}}x+1)}{x}\,dx-\cancel{{\left[\ln(-e^{-i\frac{\pi}{3}})+\ln(-e^{i\frac{\pi}{3}})\right]\int_{0}^{1}\frac{dx}{x}}} \\
        &=\int_{1}^{0} \frac{\ln(1-e^{-i\frac{\pi}{3}}x)}{x}\,dx  \int_{1}^{0} \frac{\ln(1-e^{i\frac{\pi}{3}}x)}{x}\,dx
        \end{split}
    \end{equation*}
Where the last term cancels because the coefficient of the integral is 0, as can be shown with logarithm properties. So now let's the substitution $e^{-i\frac{\pi}{3}}x\rightarrow x$ in the first integral and $e^{i\frac{\pi}{3}}x\rightarrow x$ in the second to get
\begin{equation*}
    \begin{split}
        I_2&= \int_{e^{-i\frac{\pi}{3}}}^{0} \frac{\ln(1-x)}{x}\,dx + \int_{e^{i\frac{\pi}{3}}}^{0} \frac{\ln(1-x)}{x}\,dx\\
        &= \text{Li}_2(e^{i\frac{\pi}{3}})+\text{Li}_2\left(\frac{1}{e^{i\frac{\pi}{3}}}\right)
    \end{split}
\end{equation*}
We can now apply the dilogarithm identity
\begin{equation*}
    \text{Li}_2(z)+\text{Li}_2\left(\frac{1}{z}\right)=-\frac{\pi^2}{6}-\frac{1}{2}\ln^2(-z)
\end{equation*}
\newpage So 
\begin{equation*}
    I_2=-\frac{\pi^2}{6}-\frac{1}{2}\ln^2(-e^{i\frac{\pi}{3}})=-\frac{\pi^2}{6}-\frac{1}{2}\left(-\frac{2i\pi}{3}\right)^2=\frac{\pi^2}{18}
\end{equation*}
And we have 
\begin{equation}
    \boxed{I_2=\frac{\pi^2}{18}}
\end{equation}
 Finally, we evaluate $I_3$ which is the simplest of the 3. 
 \begin{equation*}
     I_3=\int_{0}^{1} \frac{\ln{x}}{1+x}\,dx
 \end{equation*}
We will once again integrate by parts

\[
    \renewcommand{\arraystretch}{1.5}
    \begin{array}{c @{\hspace*{1.0cm}} c}\toprule
       D & I \\\cmidrule{1-2}
      +\ln{u}\tikzmark{Left 1} & \tikzmark{Right 1}\frac{1}{1+u} \\
      -\frac{1}{u} \tikzmark{Left 2} & \tikzmark{Right 2}\ln(1+u)  \\\bottomrule
    \end{array}
\]
%-----------------------------------------
\DrawArrow[draw=blue]{Left 1}{Right 2}{$ $}%
\DrawArrow[draw=red]{Left 2}{Right 2}{$ $}%
 \begin{equation*}
    \begin{split}
        I_3&= \underbrace{\ln{u}\ln(1+u)\Big|_{0}^{1}}_{=0}-\int_{0}^{1}\frac{\ln(1+u)}{u}\,du \\
        &= \int_{1}^{0} \frac{\ln(1+u)}{u}\,du
    \end{split}
\end{equation*} \newpage
Now letting $u=-x$,
\begin{equation*}
    &=\int_{-1}^{0}\frac{\ln(1-x)}{x}\,dx=\text{Li}_2(-1)=-\frac{\pi^2}{12}
\end{equation*}
And therefore
\begin{equation}
    \boxed{I_3=-\frac{\pi^2}{12}}
\end{equation}
We can now put everything together from equations (1), (2), (3), and (4) to show 
\begin{equation*}
    I=-\frac{\pi^2}{48}-\frac{2}{3}\left(\frac{\pi^2}{18}\right)+\frac{1}{3}\left(-\frac{\pi^2}{12}\right)
\end{equation*}
and we have the final result,

\begin{center}
\tcboxmath{\int_{0}^{\infty} \frac{\ln{x}}{(1+x^2)(1+x^3)}\,dx=-\frac{37 \pi^2}{432}}
\end{center}
\end{document}
