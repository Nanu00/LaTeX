\documentclass{article}
\usepackage[utf8]{inputenc}
\usepackage[utf8]{inputenc}
\usepackage[utf8]{inputenc}
\usepackage{dirtytalk}
\usepackage{bussproofs}
\usepackage{comment}
\usepackage{mathtools}
\usepackage{amsmath}
\usepackage{amsfonts}
\usepackage{amssymb}
\usepackage{indentfirst}
\usepackage{amsfonts}
\DeclarePairedDelimiter\ceil{\lceil}{\rceil}
\DeclarePairedDelimiter\floor{\lfloor}{\rfloor}
\usepackage{pgfplots}
\usepackage[utf8]{inputenc}
\usepackage{graphicx}
\usepackage[utf8]{inputenc}

\begin{document}

When looking at this problem it seems natural to draw a parallel to the single variable differential equation approach to
\begin{equation*}
    \frac{dx}{dt}=Ax, \hspace{0.1cm}x(t_0)=x_0
\end{equation*} who's solution is given by
\begin{equation*}
    x(t)=e^{(t-t_0)A}x_0
\end{equation*}
However, this same operation generalizes for differential equations of vectors (as long as the dimensions of all the matrices allows for the multiplication of course). So we also have, 
\begin{equation*}
    \frac{d\vec{x}}{dt}=A\vec{x},\hspace{0.1cm} \vec{x}(t_0)=\vec{x}_0
\end{equation*} who's solution is given by
\begin{equation*}
    \vec{x}(t)=e^{(t-t_0)A}\vec{x}_0
\end{equation*}
In this case we have $t_0=0$ so we just need to calculate
\begin{equation*}
   \vec{x}(t)=e^{At}\begin{bmatrix}
-2 \\
1  \\
4 
\end{bmatrix}
\end{equation*}
But how do we compute $e$ to a matrix?? Well, we can use the power series definition of $e^x$ to create a sum of powers of the matrix that will evaluate to it. So for a square matrix we can see
\begin{equation*}
    e^{At}=\sum_{n=0}^{\infty} \frac{t^n}{n!}A^n=I+tA+\frac{t^2}{2!}A^2+...
\end{equation*} \newpage
So it seems clear we will need a formula for any power of $A$. We will need to diagonalize A. This is because for any diagonal matrix (say 3$\times$3 in this case) we have 
\begin{equation*}
    \begin{bmatrix}
a_1 & 0 & 0\\
0 & a_2 & 0 \\
0 & 0 & a_3
\end{bmatrix}^n=  \begin{bmatrix}
a_1^n & 0 & 0\\
0 & a_2^n & 0 \\
0 & 0 & a_3^n
\end{bmatrix}
\end{equation*}
So if we write A as a product
\begin{equation*}
    A=PDP^{-1}
\end{equation*} 
for a diagonal matrix $D$, then
\begin{equation*}
    A^n=PD^nP^{-1}
\end{equation*}
Lets get to finding eigenvalues and eigenvectors. First, we set
\begin{equation*}
    \text{det}{(A-\lambda I_3)}=\begin{vmatrix}
6-\lambda & 3 & -2\\
-4 & -1-\lambda & 2 \\
13 & 9 & -3-\lambda
\end{vmatrix}=0
\end{equation*} After suitable algebra we find solutions
\begin{equation*}
    \lambda \in \{1,2,-1\}
\end{equation*}
Then, to find corresponding eigenvectors we solve the system
\begin{equation*}
    \begin{bmatrix}
6-\lambda & 3 & -2\\
-4 & -1-\lambda & 2 \\
13 & 9 & -3-\lambda
\end{bmatrix}\begin{bmatrix}
x \\
y \\
z
\end{bmatrix}=0
\end{equation*} for each value of lambda and find the corresponding eigenvectors \newpage
\begin{equation*}
    \begin{bmatrix}
1 \\
-1 \\
1
\end{bmatrix},\begin{bmatrix}
-1 \\
2 \\
1
\end{bmatrix},\begin{bmatrix}
1/2 \\
-1/2 \\
1
\end{bmatrix}
\end{equation*}
So to create a product $PDP^{-1}$ to represent $A$, we write $D$ as a diagonal matrix whose entries are the eigenvalues of $A$, $P$ as a matrix who's column vectors are the corresponding eigenvectors to $A$, and $P^{-1}$ as the inverse to this matrix. So,
\begin{equation*}
\begin{split}
    A&=PDP^{-1}\\
    e^{At}&=Pe^{Dt}P^{-1}\\
    &=
     \begin{bmatrix}
1 & -1 & 1/2\\
-1 & 2 & -1/2 \\
1 & 1 & 1
\end{bmatrix} \begin{bmatrix}
e^t & 0 & 0\\
0 & e^{2t} & 0 \\
0 & 0 & e^{-t}
\end{bmatrix} \begin{bmatrix}
5 & 3 & -1\\
1 & 1 & 0\\
-6 & -4 & 2
\end{bmatrix}\\
&=\begin{bmatrix}
5e^t-2e^{2t}-3e^{-t} & 3e^t-e^{2t}-2e^{-t} & -e^t+e^{-t}\\
-5e^t+2e^{2t}+3e^{-t} & -3e^t+2e^{2t}+2e^{-t} & e^t-e^{-t}\\
5e^t+e^{2t}-6e^{-t} & 3e^t+e^{2t}-4e^{-t} & -e^t+2e^{-t}
\end{bmatrix}
    \end{split}
\end{equation*}
Finally, since we have
\begin{equation*}
    \vec{x}(t)=e^{At}\begin{bmatrix}
-2 \\
1  \\
4 
\end{bmatrix}
\end{equation*}
we can perform the multiplication to find the final solution to the differential equation,
\begin{equation*}
    \vec{x}(t)= \begin{bmatrix}
-11e^t+e^{2t}+8e^{-t} \\
11e^t-2e^{2t}-8e^{-t}  \\
-11e^t-e^{2t}+16e^{-t} 
\end{bmatrix}
\end{equation*}
\end{document}
