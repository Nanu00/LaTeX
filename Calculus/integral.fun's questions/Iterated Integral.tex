\documentclass{article}
\usepackage[utf8]{inputenc}
\usepackage{dirtytalk}
\usepackage{bussproofs}
\usepackage{comment}
\usepackage{mathtools}
\usepackage{amsmath}
\usepackage{amsfonts}
\usepackage{amssymb}
\usepackage{indentfirst}
\DeclarePairedDelimiter\ceil{\lceil}{\rceil}
\DeclarePairedDelimiter\floor{\lfloor}{\rfloor}
\usepackage{pgfplots}
\usepackage{graphicx}
\begin{document}
\begin{equation*}
I=\int_{0}^{1}\int_{0}^{\xi_1}\int_{0}^{\xi_2}...\int_{0}^{\xi_{4999}} \ln(1-\xi_{5000})\,d\xi_{5000}...d\xi_{2}\,d\xi_{1}
\end{equation*}
One approach to this integral is to expand $\ln(1-\xi)$ as a power series, and integrate each polynomial term by interchanging summation and integration and resolving the simple sum at the end. While this is a valid approach, we will apply a slightly cleaner and more elegant solution making use of a lesser known mathematical result known as the Cauchy formula for repeated integration. It states
\begin{equation*}
    \int_{\alpha}^{x}\int_{\alpha}^{\sigma_1}...\int_{\alpha}^{\sigma_{n-1}}f(\sigma_{n})\,d\sigma_{n}...\,d\sigma_{2}\,d\sigma_1=\frac{1}{(n-1)!}\int_{\alpha}^{x}(x-t)^{n-1}f(t)\,dt
\end{equation*}
This result is major in the field of fractional calculus. I will post an inductive proof this week, but feel free to try it for yourself. In the meantime, applying to our integral,
\begin{equation*}
    I=\frac{1}{4999!}\int_{0}^{1} (1-t)^{4999}\ln(1-t)\,dt
\end{equation*}
Applying reflection identity,
\begin{equation*}
    =\frac{1}{4999!}\int_{0}^{1} t^{4999}\ln{t}\,dt
\end{equation*}
We will now apply IBP, differentiating $\ln{t}$ and integrating $t^{4999}$ to get 

\begin{equation*}
    \begin{split}
        I&=\underbrace{\frac{1}{5000!}\ln{t}\cdot  t^{5000}\Big|_{0}^{1}}_{=0}-\int_{0}^{1} \frac{t^{4999}}{5000!} \,dt\\
        &= -\frac{1}{5000!\cdot 5000}
    \end{split}
  
\end{equation*}

\end{document}
