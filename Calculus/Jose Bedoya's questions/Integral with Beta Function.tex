\documentclass{article}
\usepackage{color}
\usepackage{bigints}
\usepackage{amssymb}
\usepackage[italicdiff]{physics}
\color{white}
\definecolor{Blue}{RGB}{0,33,72}
\begin{document}
\pagecolor{Blue}
\title{Solution to Integral Fun\textsc{\char13}s Problem}
\author{Jose Bedoya}
\maketitle
\section{Question}
{\LARGE
$$I(\alpha,n)=\bigintss_{0}^{\infty} \frac{\cos \alpha t}{t^n}\,dt$$
with $\mid n \mid<1$
}
\section{Solution}
{\Large
Knowing,
$$\bigintsss_{0}^{\infty}f(t)g(t)\,dt=\bigintsss_{0}^{\infty}\mathcal{L}\{f(t)\}(s)\mathcal{L}^{-1}\{g(t)\}(s)\,ds$$

\vspace{5mm}
And by choosing
$$f(t)=\cos(\alpha t)$$
$$g(t)=\frac{1}{t^n}$$
we have
$$\mathcal{L}\{f(t)\}(s)=\frac{s}{s^2+\alpha^2}$$
$$\mathcal{L}^{-1}\{g(t)\}(s)=\frac{s^{n-1}}{\Gamma(n)}$$
\newpage
Substituting on our integral
$$I(\alpha,n)=\frac{1}{\Gamma(n)}\bigintsss_{0}^{\infty}\frac{s^n}{s^2+\alpha^2}\,ds$$

\vspace{3mm}
Let $s=\alpha\tan(\theta)$
$$I(\alpha,n)=\frac{\alpha^{n-1}}{\Gamma(n)}\bigintsss_{0}^{\frac{\pi}{2}}\tan^n(\theta)\,d\theta$$
$$=\frac{\alpha^{n-1}}{\Gamma(n)}\bigintsss_{0}^{\frac{\pi}{2}}\sin^n(\theta)\cos^{-n}(\theta)\,d\theta$$

\vspace{5mm}
Recall the definition of the Beta Function
$$B(x,y)=2\bigintsss_{0}^{\frac{\pi}{2}}\sin^{2x-1}(\theta)\cos^{2y-1}(\theta)\,d\theta$$
$$=\frac{\Gamma(x)\Gamma(y)}{\Gamma(x+y)}$$
where $\Gamma(n)=(n-1)$!.

\vspace{5mm}
Notice, in our integral:
$$x=\frac{n+1}{2}$$
$$y=\frac{1-n}{2}$$

\vspace{3mm}
Therefore,
$$I(\alpha,n)=\frac{\alpha^{n-1}}{2\Gamma(n)}\Gamma\left(\frac{n+1}{2}\right)\Gamma\left(\frac{1-n}{2}\right)$$

\vspace{3mm}
Here we can simplify the expression using the reflection property of the Gamma Function
$$\Gamma(z)\Gamma(1-z)=\frac{\pi}{\sin(\pi z)}$$
for $z$ not integer.

\vspace{3mm}
If we let $z=\frac{n+1}{2}$
$$\Gamma\left(\frac{n+1}{2}\right)\Gamma\left(\frac{1-n}{2}\right)=\frac{\pi}{\sin\left(\frac{\pi(n+1)}{2}\right)}$$
$$=\pi\sec\left(\frac{\pi n}{2}\right)$$

\vspace{2mm}
we get
}
{\LARGE
$$I(\alpha,n)=\frac{\pi}{4}\frac{\alpha^{n-1}}{\Gamma(n)}\sec\left(\frac{\pi n}{2}\right)$$
}
\end{document}