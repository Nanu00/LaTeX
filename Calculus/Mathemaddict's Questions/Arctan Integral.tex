\documentclass{article}
\usepackage[utf8]{inputenc}
\usepackage{amsmath}
\usepackage{amssymb}
\usepackage[legalpaper, portrait, margin=1in]{geometry}

\title{Mathemaddict Arctan Integral}
\author{Shreenabh Agrawal}
\date{\today}

\begin{document}

\maketitle

\section{Question}
Evaluate: $$\int\limits_{0}^{\infty} \frac{\tan ^{-1}\left(\frac{3}{2 x}\right)-\tan ^{-1}\left(\frac{1}{x}\right)}{x} \: d x$$
\section{Solution}

Taking Substitution,
$$\begin{aligned}
\frac{1}{x} &=t \\
\frac{-1}{x^{2}} \: d x &=d t
\end{aligned}$$
Hence, our Integral becomes,
$$I=\int\limits_{0}^{\infty} \frac{\tan ^{-1}\left(\frac{3 t}{2}\right)-\tan ^{-1}(t)}{t} \: dt$$
Let's derive a general formula for:-
$$I_{(a, b)}=\int_{0}^{\infty} \frac{\tan ^{-1}(a x)-\tan ^{-1}(b x)}{x} \: dx$$
Applying Leibniz Rule w.r.t Parameter $a$, 
$$\frac{\partial I_{(a, b)}}{\partial a}=\int_{0}^{\infty} \frac{1}{1+a^{2} x^{2}}\:  d x=\frac{\pi}{2 a}$$
Integrating it back, 
$$\therefore \quad I_{(a, b)}=\frac{\pi}{2} \ln (a)+c$$
If we evaluate the special case for $a=b$,
we can calculate the value of $c$ as,
$$c=-\frac{\pi}{2} \ln b$$
Hence,
$$\begin{aligned}
I_{(a, b)} 
&=\frac{\pi}{2} \ln a-\frac{\pi}{2} \ln b \\
I_{(a, b)}
&=\frac{\pi}{2} \ln \left(\frac{a}{b}\right)
\end{aligned}$$
Now substituting:-
$$a=\frac{3}{2}, b=1$$
Thus,
$$\boxed{I_{\left(\frac{3}{2}, 1\right)}=\frac{\pi}{2} \ln \frac{3}{2}}$$
\end{document}
