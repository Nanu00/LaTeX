\documentclass[12pt]{article}
\usepackage[utf8]{inputenc}
\usepackage{xcolor}
\usepackage[T1]{fontenc}
\usepackage{pagecolor}
\usepackage{amssymb}
\usepackage{lmodern}
\usepackage{mathtools, nccmath}
\usepackage{courier}
\usepackage[overload]{empheq}
\usepackage[inline, shortlabels]{enumitem}
\usepackage{amsmath}
\usepackage{mathtools} 
\definecolor{myyellow}{RGB}{225,225,100}
\definecolor{myred}{RGB}{220,100,100}
\definecolor{mygreen}{RGB}{120,225,120}
\definecolor{myblue}{RGB}{100,200,255}
\definecolor{mypurple}{RGB}{200,140,255}
\definecolor{myorange}{RGB}{255,150,50}
\color{white}
\pagecolor{black}
\title{Introduction to Linear algebra \#1}
\author{@all.about.mathematics}


\begin{document}
\maketitle
\large
\section{Introduction}
This new series of posts is designed to introduce concepts in linear algebra with some problems and solutions! 
\newline In this post, we review the very basics of linear algebra, including some \textcolor{myyellow}{properties of matrices, elementary row operations and elementary matrices}

\newpage
\section{Properties of matrices}
\textcolor{myyellow}{Theorem 1: If $A$ is invertible, then so is $adj \:A$ }
\medskip

\noindent{Proof: We start with the formula } 
\begin{equation}
A \:(adj\:A)= (\det A)\: I
\end{equation}
Since $\exists A^{-1}\iff \det A \neq 0$, we divide both sides of (1) by $\det A$
$$\left(\frac{1}{\det A}A\right)(adj \: A)=I$$
Therefore, $\left(\frac{1}{\det A}A\right)$ is the inverse of $adj \: A$.
\medskip

\noindent \textcolor{myyellow}{Theorem 2: If $\exists A^{-1}, ({A^T})^{-1}=({A^{-1}})^T$}
\medskip

\noindent{Proof: $$A^{-1} A=I\implies (A^{-1} A)^T=I$$
Since $(BC)^T=C^TB^T$, 
$$\implies A^T ({A^{-1}})^T=I\implies ({A^{-1}})^T=({A^T})^{-1}$$}

\newpage
\section{Elementary row operations}
There are 3 types of elementary row operations (EROs):
\medskip

\noindent{\textcolor{myred}{Type I: Swapping row $i$ and row $j$}}

\noindent{This is expressed as \textcolor{myred}{$\mathbf{r}_i\longleftrightarrow\mathbf{r}_j\:,\:i\neq j$}}

\noindent{The inverse of this operation is \textcolor{myred}{itself}, and applying this operation on a matrix multiplies it's determinant by \textcolor{myred}{$-1$}}
\medskip

\noindent{\textcolor{mygreen}{Type II: Multiplying row $i$ by a nonzero constant $k$}}

\noindent{This is expressed as \textcolor{mygreen}{$k\mathbf{r}_i\longrightarrow\mathbf{r}_i\:,\:k\neq 0$}}

\noindent{The inverse of this operation is \textcolor{mygreen}{$\frac{1}{k}\mathbf{r}_i\longrightarrow\mathbf{r}_i$}, and applying this operation on a matrix multiplies it's determinant by \textcolor{mygreen}{$k$}}
\medskip

\noindent{\textcolor{myblue}{Type III: Add row $i$ multiplied by $k$ to row $j$}}

\noindent{This is expressed as \textcolor{myblue}{$k\mathbf{r}_i+r_j\longrightarrow\mathbf{r}_j\:,\:k\neq 0\:,\: i\neq j$}}

\noindent{The inverse of this operation is \textcolor{myblue}{$-k\mathbf{r}_i+r_j\longrightarrow\mathbf{r}_j$}, and applying this operation on a matrix \textcolor{myblue}{does not change its determinant.}}
\medskip

\noindent{Another way to represent EROs are by \textcolor{mypurple}{elementary matrices}. Let $B$ be a matrix obtained from a square matrix $A$ by applying an ERO. Let $E$ be obtained from $I$ by \textcolor{mypurple}{applying the same ERO}. Then \textcolor{mypurple}{$B=EA$}, and $E$ is the elementary matrix representing such an ERO. }

\medskip

\noindent{\textcolor{myorange}{Important theorem:}

\noindent{We can turn square matrix $A$ into $I$ with EROs $\implies \exists A^{-1}$} }

\medskip

\noindent{Proof: Suppose we apply the EROs $e_1, e_2, \cdots, e_n$ to turn $A$ into $I$. Let $E_1, E_2, \cdots, E_n$ be the elementary matrices that represent the EROs respectively. Then we have}
$$E_n\cdots E_2 \:E_1\: A=I$$
$$\implies A^{-1}= E_n\cdots E_2 \:E_1\: $$
\end{document}

