\documentclass{amsart}
\usepackage[english]{babel}
\usepackage{graphicx}
\usepackage{amsmath}
\usepackage{amsthm}
\usepackage{amssymb}
\newtheorem{Lemma 1}{Lemma}[section]
\newtheorem{Theorem}[Lemma 1]{Theorem}
\newtheorem{Lemma 3}[Lemma 1]{Lemma}
\newtheorem{Lucas The}[Lemma 1]{Theorem}
\newtheorem{num}[Lemma 1]{Lemma}
\newtheorem{lem}[Lemma 1]{Lemma}
\newtheorem{lemme}[Lemma 1]{Lemma}
\newtheorem{lem 1}[Lemma 1]{Lemma}
\newtheorem{Main}[Lemma 1]{Theorem}
\newtheorem{Corollary}[Lemma 1]{Corollary}
\newtheorem{Corollary 1}[Lemma 1]{Corollary}
\newtheorem{Corollary 2}[Lemma 1]{Corollary}
\begin{document}
\title{On Integer sequences in Product sets}
\address{Department of Mathematics, Indian Institute of Technology Roorkee,India 247667}
\author{Sai Teja Somu}
\date{April 2 , 2017}
\maketitle
	\begin{abstract}Let $B$ be a finite set of natural numbers or complex numbers. Product set corresponding to $B$ is defined by $B.B:=\{ab:a,b\in B\}$. In this paper we give an upper bound for longest length of consecutive terms of a polynomial sequence present in a product set accurate up to a positive constant. We give a sharp bound on the maximum number of Fibonacci numbers present in a product set when $B$ is a set of natural numbers and a bound which is accurate up to a positive constant when $B$ is a set of complex numbers. 
	\end{abstract}
	\section{Introduction}
	In \cite{A} and \cite{B} Zhelezov has proved that if $B$ is a set of natural numbers then the product set corresponding to $B$ cannot contain long arithmetic progressions. In \cite{A} it was shown that the longest length of arithmetic progression is at most $O(|B|\log |B|)$. We try to generalize this result for polynomial sequences. Let $P(x)\in \mathbb{Z}[x]$ be a non constant polynomial with positive leading coefficient. Let $R$ be the longest length of consecutive terms of the polynomial sequence contained in the product set $B.B$, that is, \[R=max\{n : \text{there exists an $x\in \mathbb{N}$ such that }  \{P(x+1),\cdots,P(x+n)\}\subset B.B\}.\]We prove that $R$ cannot be large for a given non constant polynomial $P(x)$. In section 3 we consider the question of determining maximum number of Fibonacci and Lucas sequence terms in a product set. 
	Let $A\times B$ denote the cartesian product of sets $A$ and $B$.As in \cite{A} we define an auxiliary bipartite graph $G(A, B.B)$ and auxiliary graph $G'(A,B.B)$ which are  constructed for any sets $A$ and $B$ whenever $A\subset B.B$.  The vertex set of $G(A,B.B)$ is a union of two isomorphic copies of $B$ namely $B_1=B\times\{1\}$ and $B_2=B\times\{2\}$ and vertex set of $G'(A, B.B)$ is one isomorphic copy of $B$ namely $B_1=B\times \{1\}$ . For each  $a\in A$we pick a unique representation $a = b_1b_2$ where $b_1,b_2\in B$ and place an edge joining $(b_1,1)$, $(b_2,2)$ in $G(A,B.B)$ and place an edge joining  $(b_1,1)$, $(b_2,1)$ in $G'(A,B.B)$.
	Note that the number of vertices in $G(A,B.B)$ is $2|B|$ where as number of vertices in $G'(A,B.B)$ is $|B|$. Number of edges in both $G(A,B.B)$ and $G'(A,B.B)$ is $|A|$. Observe that $G'(A,B.B)$ can have self loops and $G(A,B.B)$ cannot have self loops and that $G(A,B.B)$ is necessarily a bipartite graph where as $G'(A,B.B)$ may not be a bipartite graph.
	\section{Polynomial sequences}
	We deal the problem, given a non constant polynomial $P(x)$ with positive leading coefficient and integer coefficients what can we say about the longest length of consecutive terms in the product set $B.B$.
	Since there can be at most finitely many natural numbers $r$ such that $P(r)\leq0$ or $P'(r)\leq 0$ there exists an $l$ such that $P(r+l)>0$ and $P'(r+l)>0$ for all $r\geq 1$. Hence we can assume without loss of generality that every irreducible factor $g(x)$ of $P(x)$ we have $g(x)>0 \text{ and } g'(x)>0~~ \forall x\geq 1$, as this assumption only effects $R$ by a constant. From now on we will be assuming that for every irreducible divisor $g(x)$ of $P(x)$, $g(x)>0$ and $g'(x)>0$ for all natural numbers $x$.    We prove three lemmas in order to obtain an upper bound on $R$.
	From now we let $f(x)\in \mathbb{Z}[x]$ denote an irreducible polynomial divisor of $P(x)$.  If $f(x)$ is a polynomial of degree $\geq 2$. Let $D$ be the discriminant of $f(x)$. Let $d$ be the greatest common divisor of the set $\{f(n): n\in \mathbb{N}\}$. Let $f_1(x)=\frac{f(x)}{d}$.  Denote $|D|d^2$ by $M$. If $p$ is a prime divisor of $M$ such that $p^e||M$, that is $p^e|M$ and $p^{e+1}\nmid M$,  then $p^e\nmid d$ and hence there exists an $a_p$, such that $f_1(x)$ is not divisible by $p$ for all $x\equiv a_p (\mod p^e)$.   From Chinese remainder theorem there exists an integer $a$ such that $a\equiv a_p(\mod p^e)$ for all primes $p$ dividing $M$ and hence there exists an $a$ such that $f_1(x)$ is relatively prime to $M$ for all $x\equiv a (\mod M)$. 
	\begin{num}\label{Lemma 4} For sufficiently large $R$the number of numbers in the set $\{f_1(r+i) : 1\leq i \leq R,r+i\equiv a \mod M \}$ with at least one prime factor greater than $R$ is $\geq \frac{R}{3M}$ for every non negative integer $r$. \end{num}\begin{proof}Let \[Q=\prod_{\substack{i=1\\r+i\equiv a\mod M}}^{R}f_1(r+i).\]Let $S$ be the largest divisor of $Q$ such that all the prime factors of $S$ are $\leq R$.Let $e_p$ be the index of $p$ in $S$, that is $p^{e_p}|S$ and $p^{e_p+1}\nmid S$. Let $\rho(p)$ denote the number of solutions modulo $p$ of the congruence $f(x)\equiv 0 (\mod p)$.We have,\begin{align}\log S&=\sum_{\substack{p\nmid M\\p\leq R}}e_p\log p \\&=\sum_{\substack{p\nmid M\\p\leq R}}\sum_{n=1}^{\lfloor\frac{\log f_1(r+R)}{\log p}\rfloor}\sum_{\substack{1\leq i\leq R\\r+i\equiv a\mod M\\f_1(r+i)\equiv 0 \mod p^n}}\log p.\label{key}\end{align}For a prime $p\nmid M$, as $p$ does not divide the discriminant of $f(x)$, each root $x$ of $f_1(x)\equiv 0 (\mod p)$ is a simple root modulo $p$ and each root $x$ modulo $p$ can be uniquely lifted to a solution $x'$ modulo $p^n$ of $f_1(x)\equiv 0 (\mod p^n)$. Hence number of solutions modulo $p^n$ of $f_1(x)\equiv 0(\mod p^n)$ is $\rho(p)$. From Chinese remainder theorem, each solution $x$ modulo $p^n$ such that $f_1(r+x)\equiv 0 (\mod p^n)$ with an additional congruence $r+x\equiv a (\mod M)$ corresponds to a unique solution modulo $Mp^n$. Hence number of solutions $x$ modulo $Mp^n$ such that $f_1(r+x)\equiv 0 (\mod p^n)$ and $r+x\equiv a (\mod M)$ is $\rho(p)$. From Lagrange's theorem, we have $\rho(p)\leq deg(f(x))$. Hence $\rho(p)=O(1).$
		Let $a_1,\cdots, a_{\rho(p)}$ be distinct solutions modulo $Mp^n$ of the congruences $f_1(r+x)\equiv 0 (\mod p^n)$ and $r+x\equiv a (\mod M)$. As for any $a_j$, \[\sum_{\substack{1\leq i \leq R\\i\equiv a_j(\mod Mp^n)}}1=\frac{R}{Mp^n}+O(1)\] we have,
		\begin{align*}\sum_{\substack{1\leq i \leq R\\r+i \equiv a(\mod M)\\f_1(r+i)\equiv 0(\mod p^n)}}\log p&=(\log p)\left(\sum_{j=1}^{\rho(p)}\sum_{\substack{1\leq i \leq R\\i \equiv a_j(\mod Mp^n) }}1\right)\\&=(\log p)\left( \sum_{j=1}^{\rho(p)}\frac{R}{Mp^n}+O(1)\right)\\&=\frac{R\rho(p)\log p }{Mp^n}+O(\log p).\end{align*}Combining the above result with (\ref{key}) we get \begin{align}&\log S=\sum_{\substack{p\nmid M\\p\leq R}}\sum_{n=1}^{\lfloor \frac{\log f_1(r+R)}{\log p} \rfloor}\left(\frac{R\rho(p)\log p }{Mp^n}+O(\log p)\right)\\&=\frac{R}{M}\sum_{\substack{p\nmid M\\p \leq R}}\frac{\rho(p)\log p}{p}+\frac{R}{M}\left(\sum_{\substack{p\nmid M\\p\leq R}}\sum_{n=2}^{\lfloor \frac{\log f_1(r+R)}{\log p} \rfloor} \frac{\log p}{p^n}\right)+\sum_{\substack{p\nmid M\\p\leq R}}\sum_{n=1}^{\lfloor \frac{\log f_1(r+R)}{\log p} \rfloor}O(\log p)\label{key_1}\end{align}
		
		From prime ideal theorem( See Theorem 3.2.1 of \cite{E}), we have\[ \sum_{p\leq x}\rho(p)=\text{li } x+O(xe^{-c\sqrt{\log x}})\] for some constant $c>0$. Using partial summation, we have \[ \sum_{p\leq x}\frac{\rho(p)\log p}{p}=\log x+O(1).\]Hence the first term of (\ref{key_1}) is\[\frac{R}{M}\sum_{\substack{p\nmid M\\p\leq R}}\frac{\rho(p)\log p }{p}=\frac{R\log R}{M}+O(R).\]As $\left(\sum_{\substack{p\nmid M\\p\leq R}}\sum_{n=2}^{\lfloor \frac{\log f_1(r+R)}{\log p} \rfloor} \frac{\log p}{p^n}\right)=O(1)$ the second term of (\ref{key_1}) is $O(R)$. As $\log f_1(r+R)=O(\log (r+R))$ and number of primes not dividing $M$ and less than $R$ is $O(\frac{R}{\log R})$ we have the following estimate of third term of ($\ref{key_1}$)\begin{align*}\sum_{\substack{p\nmid M\\p\leq R}}\sum_{n=1}^{\lfloor \frac{\log f_1(r+R)}{\log p} \rfloor}O(\log p)=O\left(\frac{R\log (r+R)}{\log R}\right).\end{align*}
		
		Combining the results of each term of (\ref{key_1}), we have\begin{equation}\label{equation}\log S=\frac{R\log R}{M}+O\left(\frac{\log (r+R)R}{\log R}\right). \end{equation}
		
		Let $L$ be a subset of $\{f_1(r+i) : 1\leq i\leq R,r+i\equiv a \mod M\}$ containing all the numbers which do not contain any prime factor greater than $R$ and let $l$ denote the cardinality of $L$. We have the inequality,\begin{align}\log \prod_{\substack{i=1\\f_1(r+i)\in L}}^{R}f_1(r+i)&\geq \log \prod_{i=1}^{l}f_1(r+i)\\&=n\sum_{i=1}^{l}\log (r+i)+O(l)\\&=nl\log (r+l)+O(l)\\&\geq 2l\log (r+l)+O(l)\label{key_3},\end{align}where $n\geq 2$ is the degree of the polynomial $f(x)$.Hence as $\prod_{\substack{i=1\\f_1(r+i)\in L}}^{R}f_1(r+i)|S$, we have the inequality,
		\[\log \prod_{\substack{i=1\\f_1(r+i)\in L}}^{R}f_1(r+i)\leq \log S.\] Hence from (\ref{equation}) and (\ref{key_3}) we have\begin{equation*}2l\log (r+l)+O(l)\leq \frac{R\log R}{M}+O\left(\frac{\log (r+R)R}{\log R}\right).\end{equation*}Hence for sufficiently large $R$, $l$ should be less than $ \frac{2R}{3M}-2$.  The number of numbers in the set $\{f_1(r+i) : 1\leq i \leq R,r+i\equiv a \mod M\}$ is $\geq \frac{R}{M}-1$. Hence number of numbers belonging to the set $\{f_1(r+i) : 1\leq i \leq R,r+i\equiv a \mod M\}$ with at least one prime factor greater than $R$ is $\geq \frac{R}{3M}$.\end{proof} The following corollary immediately follows from Lemma \ref{Lemma 4}. \begin{Corollary}\label{Corollary}If $P(x)\in \mathbb{Z}[x]$ has an irreducible divisor of degree $\geq 2.$ Then there exists a constant $c>0$ which may depend on $P(x)$ and independent of $R$ such that for sufficiently large $R$ and any non negative integer $r$, there are at least $cR$ numbers in the set $\{P(r+i) : 1\leq i \leq R\}$  having at least one prime factor greater than $R$. \end{Corollary}
	
	\begin{lem}\label{Lemma 5} If $f(x)$ is a linear polynomial. If $r\geq R^\gamma$ for a $\gamma >1$ then there exists a constant $c>0$ depending upon $\gamma$ such that for sufficiently large $R$, number of numbers in the set $\{f(r+i) : 1\leq i \leq R\}$ with a prime factor greater than $R$ is greater than $cR$.\end{lem}\begin{proof}The proof is similar to that of Lemma \ref{Lemma 4}.Let $Q=\prod_{i=1}^{R}f(r+i)$and $S$ be the largest divisor of $Q$ such that all the prime factors  $\leq R$, let $f(x)=sx+t$. Let $e_p$ be the index of prime $p$ dividing $S$.\begin{align}\log S&=\sum_{p\leq R}e_p\log p\\&=\sum_{p\leq R}\sum_{n=1}^{\lfloor\frac{\log f(r+R)}{\log p}\rfloor}\sum_{\substack{1\leq i\leq R\\f(r+i)\equiv 0 \mod p^n}}\log p\\&=\sum_{\substack{p\leq R\\p\nmid s}}\sum_{n=1}^{\lfloor\frac{\log f(r+R)}{\log p}\rfloor}\sum_{\substack{1\leq i\leq R\\f(r+i)\equiv 0 \mod p^n}}\log p+\sum_{\substack{p\leq R\\p| s}}\sum_{n=1}^{\lfloor\frac{\log f(r+R)}{\log p}\rfloor}\sum_{\substack{1\leq i\leq R\\f(r+i)\equiv 0 \mod p^n}}\log p. \label{key_4}\end{align}If $p\nmid s$ then $f(r+x)\equiv 0 (\mod p^n)$ has a unique solution modulo $p^n$. Let $a_1$ be the unique solution modulo $p^n$. Then \begin{align*}\sum_{\substack{1\leq i \leq R \\ f(r+i)\equiv 0 (\mod p^n)}}\log p = \sum_{\substack{1\leq i \leq R\\i\equiv a_1 (\mod p^n)}}\log p\\=\frac{R\log p}{p^n}+O(\log p).\end{align*}Observe that $\log f(r+i)= O(\log (r+i))$. Hence the first term of (\ref{key_4}) is \begin{align}\sum_{\substack{p\leq R\\p\nmid s}}\sum_{n=1}^{\lfloor\frac{\log f(r+R)}{\log p}\rfloor}\sum_{\substack{1\leq i\leq R\\f(r+i)\equiv 0 \mod p^n}}\log p=\sum_{\substack{p\leq R\\p\nmid s}}\sum_{n=1}^{\lfloor\frac{\log f(r+R)}{\log p}\rfloor}\left(\frac{R\log p}{p^n}+O(\log p)\right)\\=R\sum_{\substack{p\leq R\\p\nmid s}}\frac{\log p}{p}+R\sum_{\substack{p\leq R\\p\nmid s}}\sum_{n=2}^{\lfloor\frac{\log f(r+R)}{\log p}\rfloor}\left(\frac{\log p}{p^n}\right)+\sum_{\substack{p\leq R \\ p\nmid s}}O(\log (r+R))\label{key_5}\end{align}We have \[\sum_{\substack{p\leq R\\p\nmid s}}\frac{\log p}{p}=\log R+O(1),\] \[\sum_{\substack{p\leq R\\p\nmid s}}\sum_{n=2}^{\lfloor\frac{\log f(r+R)}{\log p}\rfloor}\left(\frac{\log p}{p^n}\right)=O(1)\] and  \[\sum_{\substack{p\leq R \\ p\nmid s}}O(\log (r+R))=O\left(\frac{R\log(r+R)}{\log R}\right).\] Substituting the above results in (\ref{key_5}) we obtain the value of first term of (\ref{key_4}) \begin{equation}\label{key_6}\sum_{\substack{p\leq R\\p\nmid s}}\sum_{n=1}^{\lfloor\frac{\log f(r+R)}{\log p}\rfloor}\sum_{\substack{1\leq i\leq R\\f(r+i)\equiv 0 \mod p^n}}\log p=R\log R+O\left(\frac{R\log(r+R)}{\log R}\right).\end{equation}If $p|s$ such that $p^k|s$ and $p^{k+1}\nmid s$. Note that $p^k\leq s$.  The number of solutions of $f(r+x)\equiv 0 (\mod p^n)$ is less than or equal to $p^k$. Let $a_1,\cdots,a_{s_1}$ be the distinct solutions modulo $p^n$. We have $|s_1|\leq p^k\leq s$ for all $n$. Note that \begin{align*}\sum_{\substack{1\leq i \leq R\\f(r+i)\equiv 0 (\mod p^n)}}\log p&= \sum_{j=1}^{s_1}\sum_{\substack{1\leq i \leq R\\ i\equiv a_j (\mod p^n)}}\log p\\&=\sum_{j=1}^{s_1}(\log p)\left(\frac{R}{p^n}+O(1)\right)\\&=s_1\left( \frac{R\log p}{p^n} +O(\log p)\right)\\&\leq s\left( \frac{R\log p}{p^n} +O(\log p)\right).\end{align*}Using the above result we obtain an estimate on second term of (\ref{key_4})\begin{align}\sum_{\substack{p\leq R\\p| s}}\sum_{n=1}^{\lfloor\frac{\log f(r+R)}{\log p}\rfloor}\sum_{\substack{1\leq i\leq R\\f(r+i)\equiv 0 \mod p^n}}\log p&\leq \sum_{\substack{p\leq R\\p|s}}\sum_{n=1}^{\lfloor \frac{\log f(r+R)}{\log p}\rfloor}s\left(\frac{R \log p}{p^n}+O(\log p)\right).\label{second}\end{align}As there are only $O(1)$ number of prime factors of $s$ and $\log f(r+R)=O(\log (r+R)) $ we have  \[ \sum_{\substack{p\leq R\\p|s}}\sum_{n=1}^{\lfloor \frac{\log f(r+R)}{\log p}\rfloor}s\left(\frac{R \log p}{p^n}\right)=O(R)\] and \[ \sum_{\substack{p\leq R\\p|s}}\sum_{n=1}^{\lfloor \frac{\log f(r+R)}{\log p}\rfloor} O(s\log p)=O(\log (r+R)).\]Combining the above two results in (\ref{second}) we have an estimate for the second term of (\ref{key_4})\begin{equation}\label{second_term}\sum_{\substack{p\leq R\\p| s}}\sum_{n=1}^{\lfloor\frac{\log f(r+R)}{\log p}\rfloor}\sum_{\substack{1\leq i\leq R\\f(r+i)\equiv 0 \mod p^n}}\log p=O(R+\log (r+R)).\end{equation}From (\ref{key_4}), (\ref{key_6}) and (\ref{second_term}) we have \begin{equation}\label{main}\log S=R\log R+O\left(\frac{R\log (r+R)}{\log R}\right).\end{equation} Let $L$ be a subset of \{$1\leq i\leq R$\} containing all $i$ such that $f(r+i)$ has all prime factors $\leq R$. Let the cardinality of $L$ be $l$.\begin{align*}\log \prod_{\substack{i=1\\i\in L}}^{R}f(r+i)&\geq \log \prod_{i=1}^{l}f(r+i)\\&=l\log (r+R) +O(R). \end{align*}As $\prod_{\substack{i=1\\i\in L}}^{R}f(r+i)|S$ we have $\log \prod_{\substack{i=1\\i\in L}}^{R}f(r+i)\leq \log S$. From (\ref{main}) and the above inequality we have \begin{equation*} l \log (r+R)+O(R)\leq R\log R +O\left(\frac{R\log(r+R)}{\log R}\right).\end{equation*}For sufficiently large $R$, $l$ should be $\leq \frac{(1+\gamma)}{2\gamma}R$. Hence for sufficiently large $R$ number of numbers of the set $\{f(r+i) : 1\leq i \leq R  \}$ with at least one prime factor greater than $R$ is $\geq \frac{(\gamma-1)R}{2\gamma }$.\end{proof}We have the following Corollary for Lemma \ref{Lemma 5}.\begin{Corollary 1}\label{Corollary 2}If degree of every irreducible divisor of $P(x)$ is 1 and $\gamma>1$ then there exists a positive constant $c$ such that the number of elements of the set  $\{P(r+i): 1\leq i \leq R\}$ having at least one prime factor greater than $R$ is greater than $cR$ for sufficiently large $R,r$ satisfying the inequality $r\geq R^\gamma$.\end{Corollary 1}
	
	\begin{lemme}\label{Lemma 6}Let $f(x)$ be a linear polynomial.If $r\leq R^\gamma$ for some $\gamma>1$ then there are at least $c\frac{R}{\log R}$ numbers of the set $\{f(r+i): 1\leq i \leq R \}$ with at least one prime factor greater than $\frac{R}{2}$ for a constant $c>0$ and sufficiently large $R$. \end{lemme}\begin{proof}Let $f(n)=sn+t$ then there are at least $c_1(\frac{R}{\log R})$ primes between $(\frac{R}{2},R]$ which are coprime to $s$, for a constant $c_1>0$. Each prime in the interval $(\frac{R}{2},R]$ has one or two $i\in [1,R]$ such that $p|f(r+i)$. For each $f(r+i)$ there are at most $O(1)$ prime divisors belonging to $(\frac{R}{2},R]$. Hence there are at least $c\frac{R}{\log R}$ numbers with at least one prime factor greater than $\frac{R}{2}$ for sufficiently large $R$ and for some constant $c>0$.\end{proof}
	\begin{Corollary 2}\label{Corollary 3}If degree of every irreducible divisor of $P(x)$ is 1 and $r\leq R^\gamma$ then number of elements of the set  $\{P(r+i): 1\leq i \leq R\}$ with at least one prime factor belonging to the range $(\frac{R}{2},R]$ is greater than $c\frac{R}{\log R}$ for a constant $c>0$ and for sufficiently large $R$.\end{Corollary 2}
	In a graph $G(V,E)$ for $v\in V$ we define $V(v)$ to be the set of all vertices adjacent to $v$. Now we require a graph theoretic result in order to obtain an upper bound on $R$.
	\begin{lem 1}\label{Lemma 7}If there is a bipartite graph $(A,B,E)$ such that for all $a\in A$ and $b\in B$, degree of $a$ is $\leq n$ and degree of $b$ is $\geq 1$ then there exists a sequence of vertices $b_1,\cdots,b_k$ with $b_i\in B$ satisfying $V(b_1)\neq \phi$ and $V(b_i)/(\cup_{j=1}^{j=i-1}V(b_j))\neq \phi$ for $2\leq i\leq k$ and $k\geq \frac{|B|}{n}$.\end{lem 1}\begin{proof}The proof is by induction on $n$. For $n=1$ the lemma is true since degree of $a\leq 1~~ \forall ~ a\in A\implies V(b_1)\cap V(b_2)=\phi ~~\forall~ b_1\neq b_2\in B$ and the sequence $b_1,\cdots, b_{|B|}$ will clearly satisfy $V(b_1)\neq \phi$ and $V(b_i)/(\cup_{j=1}^{i-1}V(b_j))\neq \phi$ for  $2\leq i\leq k$.If the lemma is true for $n=r$ we have to prove for $n=r+1$. Order the vertices of $B$ as $b_1,\cdots b_{|B|}$. Let $S=\{a\in A : \text{degree of $a$}\geq 1\}$.Let $S_1=V(b_1)$ and for $2\leq i \leq |B|$, let  $S_i=V(b_i)/(\cup_{j=1}^{i-1}V(b_j))$.Observe that $S=\cup_{i=1}^{|B|}S_i$.Let $K$ be a set defined by $K=\{b_i : S_i\neq \phi\}$.If $|K|\geq \frac{|B|}{r+1}$ then we can choose the vertices in the set $K$ arranged in a sequence which satisfies the hypothesis.If $|K|<\frac{|B|}{r+1}$ then consider the induced subgraph $A\cup (B/K)$ then degree of $a$ is less than or equal to $r$ for all $a\in A.$ From the induction assumption there exists a sequence with length $\geq \frac{|B/K|}{r}>|B|(1-\frac{1}{r+1})\frac{1}{r}=\frac{|B|}{r+1}$ in $B/K$ satisfying the hypothesis which completes the proof by induction.   \end{proof}Now we prove the main theorem.\begin{Main}If $P(x)\in \mathbb{Z}[x]$ is a non constant polynomial with a positive leading coefficient and $B$ is a set of complex numbers. If $\{P(r+1),\cdots,P(r+R)\}$ is contained in the product set $B.B$ for a nonnegative integer $r$ and a natural number $R$. Then  \newline(1)If $P(x)$ has an irreducible factor of degree $\geq 2$ then $R=O(|B|)$. In this case the implicit constant only depends on $P(x)$.\newline(2)If $P(x)$ has no irreducible factor of degree $\geq 2$ and there exists a $\gamma>1$ such that $r>R^\gamma$ then $R=O(|B|)$. In this case the implicit constant depends on $P(x)$ and $\gamma$. \newline(3)If $P(x)$ has no irreducible factor of degree $\geq 2$ and there exists a $\gamma>1$ such that $r\leq R^\gamma$ then $R=O(|B|\log |B|)$. In this case the implicit constant depends on $\gamma$ and $P(x)$.\end{Main}\begin{proof}If $P(x)$ has an irreducible factor $f(x)$ of degree  $\geq2$ or $P(x)$ has no irreducible divisor of degree $\geq 2$ and $r> R^\gamma$ for some $\gamma>1$ let \begin{equation*}A=\{p : \text{$p$ is a prime,~$p|P(r+i)$ for some $1\leq i\leq R$,~$p>R$}\}\end{equation*}and let\begin{equation*}C=\{P(r+i) : \text{$1\leq i \leq R$,$\exists$ prime $p>R$ \text{ such that }  $p|P(r+i)$}\}.\end{equation*}If $P(x)$ has no irreducible divisor of degree $\geq 2$ and $r\leq R^\gamma$ for some $\gamma>1$ then let \begin{equation*}A=\{p : \text{$p$ is a prime, $\frac{R}{2}< p\leq R$ and $p|P(r+i)$ for some $1\leq i \leq R$}\}\end{equation*}and let \begin{equation*}C=\{P(r+i) : \text{$1\leq i \leq R$,$\exists$ prime $p\in (\frac{R}{2},R]$ \text{ such that } $p|P(r+i)$}\}.\end{equation*}In cases (1) and (2) from Corollaries \ref{Corollary} and \ref{Corollary 2} the size of $C$ is greater than $cR$ for some constant $c>0$ ($c$ depends only on $P(x)$ in case (1) and depends on $P(x)$ and $\gamma$ in case (2)) and for sufficiently large $R$. In case (3) from Corollary \ref{Corollary 3} the size of $C$ is greater than $c\frac{R}{\log R}$ for sufficiently large $R$ and for some constant $c>0$ which depends on $\gamma$ and $P(x)$.If we consider a bipartite graph $G$ between $A\cup C$ constructed such that there exits an edge $p\in A$ and $P(r+i)\in C$ if and only if $p|P(r+i)$. Let degree of $P(x)$ be $d$. In this graph, from Lagrange's theorem the degree of $a\in A$ is less than or equal to the degree of polynomial $P(x)$ in case (1) and (2) and less than twice the degree of polynomial $P(x)$ in case (3) . Hence from Lemma \ref{Lemma 7} there exists a sequence $c_1,c_2,\cdots ,c_k\in C$ with $k\geq \frac{|C|}{2d}$ such that $V(c_1)\neq \phi$ and $V(c_i)/\cup_{j=1}^{k-1}V(c_j)\neq \phi$.  Therefore every $c_i$ has a prime divisor which does not divide any of $c_j$ for $1\leq j \leq i-1$. Let  $C'=\{c_1,\cdots,c_k\}$. Note that in cases (1) and (2) we have $|C'|\geq \frac{c}{2d}R$  and in case (3) we have $|C'|\geq \frac{cR}{2d\log R}$ for sufficiently large $R$.
		Consider the bipartite auxiliary graph $G(C',B.B)$. We claim that there cannot be any cycle in this graph. Note that the vertex set of $G(C',B.B)$ is $B\times\{1\}\cup B\times\{2\}$ Suppose there was a cycle, since it is a bipartite graph, cycle length has to be an even number and there will be a cycle of the form  $(b_1,1)(b_2,2)\cdots (b_{2k},2)(b_1,1)$ where $b_i's$ belong to $B$ then \begin{align*}  b_1b_2&=c_{n_1}\\  b_2b_3&=c_{n_2}\\        &.       \\        &.\\        &.\\  b_{2k}b_1&=c_{n_{2k}}    \end{align*}for some $n_i$'s such that $1\leq n_i\leq k$ and $n_i\neq n_j$ for $i\neq j$. Then it is easy to observe the relation \begin{equation}\label{contradiction}\prod_{i=1}^{k}c_{n_{2i}}=\prod_{j=1}^{k}c_{n_{2j-1}}\end{equation}let $n_i$ be the largest element of the set $\{n_1, \cdots,n_{2k}\}$. There exists a prime $p$ such that $p|c_{n_i}$ and $p\nmid c_{n_j}$ for $j\neq i$ and hence $p$ divides exactly one side of (\ref{contradiction}) and hence (\ref{contradiction}) cannot be true. Thus there exists no cycle in $G(C',B.B)$. In any graph without self loops, if there exists no cycle then the numbers of edges is strictly less than number of vertices. Hence $|C'|\leq 2|B|-1$.  As in cases (1) and (2) we have $|C'|\geq \frac{c}{2d}R$  and in case (3) we have $|C'|\geq \frac{cR}{2d\log R}$ for sufficiently large $R$.Therefore $R=O(|B|)$ in cases (1),(2) and $R=O(|B|\log |B|)$ in case (3) which completes the proof of the theorem. \end{proof}
	\section{Number of Fibonacci Numbers and Lucas Numbers in a Product set}
	Let $B$ be a finite set of naural numbers. Let $A$ be the set of Fibonacci numbers contained in the product set $B.B$. From \cite{C} there are only two perfect square Fibonacci numbers, viz., $1$ and $144$. Hence there can be at most two self loops in the  graph $G'(A,B.B)$. We give an upper bound on the cardinality of $A$ by using the following lemma.
	\begin{Lemma 1}\label{Lemma}Let $F_n$ and $F_m$ be $n$th and $m$th Fibonacci numbers and $m<n$ and $n>2$ then $gcd(F_n,F_m)<\sqrt{F_n}$.\end{Lemma 1}\begin{proof}Let $d=gcd(m,n)$. From the strong divisibilty property of Fibonacci numbers $gcd(F_n,F_m)=F_{d}$. We know that $F_n=\frac{\alpha^n-\beta^n}{\alpha-\beta},$ where $\alpha=\frac{1+\sqrt{5}}{2}$ and $\beta=\frac{1-\sqrt{5}}{2}$. Since $m<n$, clearly $d\leq \frac{n}{2}$.  If $d=1$ then the hypothesis is clearly true and if $d>1$, we have\begin{equation*}(F_d)^2=\frac{(\alpha^d-\beta^d)^2}{(\alpha-\beta)^2}< \frac{(\alpha^{2d}-\beta^{2d})}{(\alpha-\beta)}\leq F_n.\end{equation*}Thus, $gcd(F_m,F_n)<\sqrt{F_n}$.\end{proof}
	\begin{Theorem}There cannot be more than $|B|$ Fibonacci numbers in the product set $B.B$ when $B$ is a set of natural numbers.\end{Theorem}\begin{proof}We claim that in the graph $G'(A,B.B)$ there cannot any cycle other than self loops. Suppose there is a $k$-cycle of the form $(b_1,1)(b_2,1)\cdots (b_k,1)(b_1,1)$ where $b_i\in B$ for $1\leq i \leq k$. We now have the information that $b_ib_{i+1}$ for $1\leq i \leq k-1$ and $b_kb_1$ are distinct Fibonacci numbers in the set $B.B$. Without loss of generality let us assume $b_1b_2=F$ is the largest Fibonacci number among $b_ib_{i+1}$ for $1\leq i \leq k-1$ and $b_kb_1$.From Lemma \ref{Lemma}, we have \begin{align*}& b_1\leq gcd(b_1b_2,b_1b_k)<\sqrt{F},\\& b_2\leq gcd(b_1b_2,b_2b_3)<\sqrt{F}.\end{align*}Hence $F=b_1b_2<F$ which is a contradiction.Hence there cannot be any cycle. From \cite{C} there cannot be more than 2 self loops.Hence the number of edges which equal number of Fibonacci numbers in the set $B.B$ cannot exceed $|B|+1$.
		Now we prove that there cannot be $|B|+1$ Fibonacci numbers in $B.B$. Suppose there are $|B|+1$ Fibonacci numbers, as the graph cannot have any cycle there should be two self loops namely, $(1,1)$ and $(12,1)$, and $1,12\in B$ and the graph obtained by removing the two self loops should be connected tree of $|B|$ vertices. Since the graph is connected there should be a path between $(1,1)$ and $(12,1)$. Let the path be $(b_1,1)(b_2,1)\cdots (b_k,1)$ which implies that  $b_ib_{i+1}$ for $1\leq i\leq k-1$ are Fibonacci numbers and without loss of generality assume $b_1=1$ and $b_k=12$. Let $l$ be the index of highest value of $b_ib_{i+1}$, that is $b_lb_{l+1}$ is the largest Fibonacci number among $b_ib_{i+1}$ for $1\leq i\leq k-1$. Clearly $l\neq 1$ and if $2\leq l\leq k-2$ then from Lemma \ref{Lemma} we have  \begin{align*}& b_l\leq gcd(b_lb_{l+1},b_{l-1}b_l)<\sqrt{b_lb_{l+1}},\\& b_{l+1}\leq gcd(b_lb_{l+1},b_{l+1}b_{l+2})<\sqrt{b_lb_{l+1}}.\end{align*}Which implies $b_lb_{l+1}<b_lb_{l+1}$. Hence $l=k-1$. Again from Lemma \ref{Lemma}\begin{equation*}b_{k-1}\leq gcd(b_{k-1}b_k,b_{k-2}b_{k-1})<\sqrt{b_{k-1}b_k}\end{equation*}which implies $b_{k-1}<b_k=12$ but there are no Fibonacci numbers of the form $12b$ with $b<12$.Hence there cannot be $|B|+1$ Fibonacci numbers. Thus number of Fibonacci numbers in the set $B.B$ is $\leq |B|$. \end{proof}Now we consider the case where $B$ is a set of complex numbers and try to give an upper bound on the number of Lucas sequence terms in the product set. Let $A$ be the set of Lucas sequence terms with indices greater than $30$ in the  product set $B.B$.   
	\begin{Lemma 3}\label{Lemma 3}There cannot be any cycle in $G(A,B.B)$.\end{Lemma 3}\begin{proof}Suppose there was a cycle,  then there will be a cycle of the form $(b_1,1)(b_2,2)\cdots (b_{2k},2)(b_1,1)$.Then \begin{align*}  b_1b_2&=L_{n_1}\\  b_2b_3&=L_{n_2}\\        &.       \\        &.\\        &.\\  b_{2k}b_1&=L_{n_{2k}},      \end{align*}where $L_{n_i}$ are Lucas sequence terms with indices greater than $30$, which implies\begin{equation}\label{Lucas}\prod_{i=1}^{k}L_{n_{2i}}=\prod_{j=1}^{k}L_{n_{2j-1}}\end{equation}Let $n_i$ be the largest index $\geq 31$. Then from \cite{D}, $L_{n_i}$ contains a primitive divisor $p$ and hence $p$ divides exactly one side of (\ref{Lucas}) and therefore (\ref{Lucas}) cannot be true. Thus there cannot be any cycle.\end{proof}
	\begin{Lucas The}Let $(L_n)_{n=1}^\infty$ be a Lucas sequence. Then the number of distinct elements of $(L_n)_{n=1}^\infty$ in $B.B$ is less than $2|B|+30$.\end{Lucas The}\begin{proof}Since the number of vertices in $G(A,B.B)$ is $2|B|$ and from Lemma \ref{Lemma 3} there cannot be a cycle in $G(A,B.B)$ and hence the number of edges in $G(A,B.B)$ is $\leq 2|B|-1$. Hence the number of distinct terms in Lucas sequence of index $\geq 31$ is $\leq 2|B|-1$. Hence number of distinct Lucas sequence terms in $B.B$ is $\leq 2|B|+29$.\end{proof}
	
	\bibliographystyle{amsplain}
	
	\begin{thebibliography}{10}
		
		\bibitem{D}Y. Bilu, G. Hanrot and P. Voutier, {\it  Existence of primitive divisors of Lucas and Lehmer numbers.} J. Reine Angew. Math. {\bf 539}(2001), 75-122.
		    
	   \bibitem{C}Y. Bugeaud, M. Mignotte, and S. Siksek, {\it Classical and modular approaches  to exponential Diophantine equations. I. Fibonacci and Lucas perfect powers}, Ann. of Math. (2)  {\bf 163}:3 (2006), 969-1018.
	        
	   \bibitem{E} Cojocaru, A. C., Murty, M. R. (2005). {\it An introduction to sieve methods and their applications} (Vol. 66). Cambridge University Press.
	          
	   \bibitem {A} D. Zhelezov, {\it Improved bounds for arithmetic progressions in product sets},   International Journal of Number Theory 11, no. 08(2015), 2295-2303.
	        
	   \bibitem {B}  D. Zhelezov, {\it Product sets cannot contain long arithmetic progressions}, Acta Arith. 163    (2014), 299-307.         
   \end{thebibliography}
\end{document}